\documentclass[a4j]{jarticle}
\usepackage[dvipdfmx]{graphicx}
\usepackage{float}
\usepackage{array,booktabs}
\usepackage{subcaption}
\usepackage{itembkbx}
\usepackage{eclbkbox}
\usepackage{moreverb}
\usepackage{url}
\usepackage{amsmath,amssymb}
\usepackage{eqnarray}
\usepackage{multirow}
\usepackage{fancybox}
\usepackage{ascmac}
\usepackage[newitem, newenum]{paralist}
\usepackage[top=25.0truemm, bottom=30.0truemm, left=26.0truemm, right=27.5truemm]{geometry}
\usepackage{listings,jlisting}
\usepackage{lscape}
\lstset{
language={C++},
basicstyle={\small},
identifierstyle={\small},
commentstyle={\small\itshape},
keywordstyle={\small\bfseries},
ndkeywordstyle={\small},
stringstyle={\small\ttfamily},
frame={tbrl},
breaklines=true,
breakindent = 8.58pt,
columns=[l]{fullflexible},
numbers=left,
xrightmargin=0zw,
xleftmargin=3zw,
numberstyle={\scriptsize},
stepnumber=1,
numbersep=1zw,
lineskip=-0.0ex,
showstringspaces=false
}
\title {自主課題研究\\中間レポート\\}
\author {3年 244 清水翔仁 \\学籍番号:1852130904}
\date{\西暦 提出日:\today}
\begin{document}
  %タイトル
  \maketitle
  \thispagestyle{empty}
  \newpage
  \setcounter{page}{1}
  \pagestyle{plain}

\section{研究内容}
  \subsection{テーマ名}
    RNNを使ったPOMDP環境でのCartPole
  \subsection{概要}
    POMDP環境において,DRQNやR2D2などいくつかの手法でCartPoleを実装しそれぞれの結果の比較・検証を行う.

\section{研究計画}
  今後の研究計画を表\ref{tab_plan}に示す.
    \begin{table}[!h]
      \centering
      \caption{研究計画}
      \label{tab_plan}
      \begin{tabular}{l|l}
      \hline
      2020/11/20 & プログラムの改善,結果の記録 \\
      \hline
      2020/11/27 & プログラムの解析,必要な知識の学習 \\
      \hline
      2020/12/11 & プログラムの解析,必要な知識の学習 \\
      \hline
      2020/12/18 & 結果の検証・考察 \\
      \hline
      2020/12/25 & 結果の検証・考察 \\
      \hline
      2021/01/08 & 発表練習,最終レポート作成 \\
      \hline
      2021/01/13 & 発表練習,最終レポート作成 \\
      \hline
      2021/01/22 & 発表会 \\
      \hline
      \end{tabular}
    \end{table}

\section{進捗状況}
  本研究に必要な基礎知識として,機械学習,deeplearning,強化学習についての学習を行った.github上にあるプログラムを参考にして実装・検証を行う予定だが,現在実行時にエラーが発生してしまうため,原因を調査中.

\section{今後の方針}
  表\ref{tab_plan}に示す研究計画に基づき,まずはプログラムのバグ修正を行う.その後正しい実行結果を記録し,ソースコードと照らし合わせそれぞれの手法の理解を深める.特にRNNやPOMDPに関してはかなり専門的な内容になるため,より長期間の学習が必要であると考えられる.またプログラムの実行結果の検証・考察を年内に完了させ,来年は発表資料の作成,最終レポートに研究結果をまとめることに集中する.

\end{document}
