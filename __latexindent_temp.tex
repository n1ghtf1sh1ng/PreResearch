\documentclass[a4j]{jarticle}
\usepackage[dvipdfmx]{graphicx}
\usepackage{float}
\usepackage{array,booktabs}
\usepackage{subcaption}
\usepackage{itembkbx}
\usepackage{eclbkbox}
\usepackage{moreverb}
\usepackage{url}
\usepackage{amsmath,amssymb}
\usepackage{eqnarray}
\usepackage{multirow}
\usepackage{fancybox}
\usepackage{ascmac}
\usepackage[newitem, newenum]{paralist}
\usepackage[top=25.0truemm, bottom=30.0truemm, left=26.0truemm, right=27.5truemm]{geometry}
\usepackage{listings,jlisting}
\usepackage{lscape}
\lstset{
language={C++},
basicstyle={\small},
identifierstyle={\small},
commentstyle={\small\itshape},
keywordstyle={\small\bfseries},
ndkeywordstyle={\small},
stringstyle={\small\ttfamily},
frame={tbrl},
breaklines=true,
breakindent = 8.58pt,
columns=[l]{fullflexible},
numbers=left,
xrightmargin=0zw,
xleftmargin=3zw,
numberstyle={\scriptsize},
stepnumber=1,
numbersep=1zw,
lineskip=-0.0ex,
showstringspaces=false
} 
\title {自主課題研究\\最終レポート\\}
\author {3年 244 清水翔仁 \\学籍番号:1852130904}
\date{\西暦 提出日:\today}
\begin{document}
  %タイトル
  \maketitle
  \thispagestyle{empty}
  \newpage
  \setcounter{page}{1}
  \pagestyle{plain}

% 目的,課題内容,実行環境
\section{目的}

\section{課題内容} 
  %\begin{itemize}
  %\setlength{\parindent}{1zw}
  %\end{itemize}

\section{実行環境}
  本実験の実行環境を表\ref{tab_env}に示す.
  %表1
  \begin{table}[htbp]
    \centering 
    \caption{実験の実行環境}
    \label{tab_env}
      \begin{tabular}{c|l}
        \toprule
        CPU & Intel(R) Core(TM) i7-5650U CPU @ 2.20GHz \\
        メモリ容量 & 8.00 GB \\
        OS & Windows 10 Education \\
        統合開発環境 & Microsoft Visual Studio \\
        \bottomrule
      \end{tabular}
  \end{table}

% プログラムの説明,実行結果
\section{プログラムの説明}

\section{実行結果}\label{result}
  %\begin{figure}[H]
  %\centering
  %\includegraphics[width=8cm]{./images/start.pdf}
  %\caption{起動時の画面}
  %\label{fig_start}
  %\end{figure}

% 考察,参考文献
\section{考察}

\begin{thebibliography}{9}
  \bibitem{1} Reference
\end{thebibliography}

% 付録(ソースコード)
\appendix
\def\thesection{付録\Alph{section}}\section{プログラムリスト}\label{appendix}
  本実験で作成したソースコードをソースコード\ref{code}に示す.

\begin{lstlisting}[label = code, caption = CaptionTitle]

\end{lstlisting}
\end{document}