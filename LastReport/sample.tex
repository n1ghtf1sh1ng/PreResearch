%%%%%%%%%%%%%%%%%%%%%%%%%%%%%%%%%%%%%%%%%%%%%%%%%%%%%%%%%%%%%%%%%%%%%%
%   長野高専電子情報工学科 卒業論文テンプレート
%
%   Original by Takuma Yoshida, 2010
%   Modified by Shoichi Ito, 2014-
%
%%%%%%%%%%%%%%%%%%%%%%%%%%%%%%%%%%%%%%%%%%%%%%%%%%%%%%%%%%%%%%%%%%%%%%

% 卒業論文スタイルファイルの適用([a4,11pt]のようなオプションは不要)
\documentclass{eithesis}

% よくつかうパッケージの読み込み
\usepackage{verbatim}
\usepackage{fancybox}
\usepackage{subcaption}
\usepackage{graphicx}
\usepackage{courier}
\usepackage{url}
\usepackage{fancyhdr}

\begin{document}
%%%%%%%%%%%%%%%%%%%%%%%%%%%%%%%%%%%%%%%%%%%%%%%%%%%%%%%%%%%%%%%%%%%%%%
%   表紙の設定
%%%%%%%%%%%%%%%%%%%%%%%%%%%%%%%%%%%%%%%%%%%%%%%%%%%%%%%%%%%%%%%%%%%%%%
\etGengou{平成98年度}      % 年度
\etTitle{標準模型における自発的対称性の破れ}     % 論文タイトル(日本語)
\etTitleEn{Spontaneously Symmetry Breaking in the Standard Model}    % 論文タイトル(英語)
\etDate{平成99年1月31日}   % 提出日(1月以降は年に注意)
\etAuthor{福山 雅治}       % 著者フルネーム
\etLabName{湯川研究室}     % 研究室名
\etMyProfessor{湯川 学}    % 指導教員フルネーム
\etMakeTitle
%%%%%%%%%%%%%%%%%%%%%%%%%%%%%%%%%%%%%%%%%%%%%%%%%%%%%%%%%%%%%%%%%%%%%%
%   目次の出力
%%%%%%%%%%%%%%%%%%%%%%%%%%%%%%%%%%%%%%%%%%%%%%%%%%%%%%%%%%%%%%%%%%%%%%
\pagenumbering{roman}
\tableofcontents
\clearpage
\pagenumbering{arabic}
%%%%%%%%%%%%%%%%%%%%%%%%%%%%%%%%%%%%%%%%%%%%%%%%%%%%%%%%%%%%%%%%%%%%%%
%   ヘッダー・フッターの設定
%%%%%%%%%%%%%%%%%%%%%%%%%%%%%%%%%%%%%%%%%%%%%%%%%%%%%%%%%%%%%%%%%%%%%%
\pagestyle{fancyplain}
\lhead{\leftmark}  % ヘッダー左側(\leftmarkのデフォルトはsectionhead)
\chead{}           % ヘッダー中央
\rhead{\rightmark} % ヘッダー右側(\rightmarkのデフォルトはchapterhead)
\lfoot{}           % フッター左側
\cfoot{\thepage}   % フッター中央(ページ番号)
\rfoot{}           % フッター右側

%%%%%%%%%%%%%%%%%%%%%%%%%%%%%%%%%%%%%%%%%%%%%%%%%%%%%%%%%%%%%%%%%%%%%%
%   論文本体
%%%%%%%%%%%%%%%%%%%%%%%%%%%%%%%%%%%%%%%%%%%%%%%%%%%%%%%%%%%%%%%%%%%%%%
\chapter{はじめに}
ここで,論文で採りあげていることについて,
現状・問題意識・それが解決するとどんないいことがあるか
などについてデータを並べながら説明する.
最後に,論文全体の構成(2章では○○について述べ,3章では・・・)を書く.
全体で1〜2ページ.
\clearpage


\chapter{予備知識いろいろ}
章の最初には,その章を簡潔にまとめた数行のリード文を入れ,
その章を読むべきかどうかの判断材料を読者に与える.

\section{セクション}
「はじめに」の次は論文を読むための予備知識の章である.
いくつかのトピックスについて個々の章とするか,すべて
まとめて1つの章とするかは分量による.
類似の製品などについても,概要・何が違うか・メリット・デメリット・
応用製品,などについてデータや写真を交えながら詳細に書く.

\subsection{サブセクション}
この例のように,セクションの中にサブセクションが1つだけであれば
わざわざそれらを作らない.

\section{test1}
\section{test2}
\clearpage
\section{test3}
\section{test4}
\clearpage
\section{test5}
\section{test6}
\clearpage

\chapter{作ったものの解説}
自分が作ったものの解説.
取扱説明書ではないのだから,内部の詳細な実装方法や
なぜそのアルゴリズム(部品)を採用したのかなどに
ついてもしっかり記述する.
自分の後輩がこの論文だけを頼りに,この論文に記述されている実験結果を
すべて後輩のPCで再現することができて,この論文の著者が不在でも
きちんと研究の引き継ぎができるように書く.

プログラムリストはこのように入れる.
これはソースコード\ref{fuga}です.プログラムリストの
文字コードを\LaTeX ソースの文字コードとあわせておくことに注意.
一貫してUTF-8+LFで書いておくのがよいだろう.
見た目はeithesis.clsの\verb+\lstset+のところで変えられるほか,
eithesis.clsで設定した内容をこの\LaTeX ファイル内で上書きしても良い.

\lstinputlisting[caption=hoge, label=fuga]{program.c}


\chapter{評価}
作ったものの客観的な評価を書く.単に一度だけ使ってみて
「よさそうです」とか,「友達数人に使ってみてもらったところ
『便利です』と言っていました」は評価とはいわない.
被験者一人が一度だけ使ってみていったい何が判るというのだろうか.

一回プログラムを走らせて取っただけのデータに信頼性があるか?
確率統計に立脚したデータ処理をきちんと行い,第三者から
「それって,たまたまじゃないの?」と言われないようにする.

いったん評価をしたものの,自分の主張を読者に納得してもらうためには
まったくデータ不足であることが判明することが非常に多い.
従って,研究の中盤から早々に評価フェーズに入り,評価して改良して
また評価,というサイクルを繰り返すことが必要である.これによって,
「はじめに」で振った問題意識が自分の作ったシステムで確かに
解決されているということをデータで納得させる.


\chapter{まとめ}
「はじめに」で振った話や問題意識がどれだけ回収できているか,
なにが問題として残ったのか.
あらためて研究のはじめから終わりまでの全体を俯瞰してのまとめを書く.
どうせやる予定のない「今後の予定」は書いてはいけない.


%%%%%%%%%%%%%%%%%%%%%%%%%%%%%%%%%%%%%%%%%%%%%%%%%%%%%%%%%%%%%%%%%%%%%%
%   謝辞と参考文献
%%%%%%%%%%%%%%%%%%%%%%%%%%%%%%%%%%%%%%%%%%%%%%%%%%%%%%%%%%%%%%%%%%%%%%
\chapter*{謝辞}
ここは自由に書いて良い.その人の協力なくしてこの研究は成し遂げられなかった
と思われる人への謝意をあらわす.名前は基本的にフルネームで入れる.

\begin{thebibliography}{99}
\bibitem{hoge1}{ビジュアライジング・データ: Ben Fry(著), 増井俊之(監訳), 加藤慶彦(訳), 
オライリー・ジャパン, 2008.}
\bibitem{hoge2}{長野市の天気 - Yahoo!天気・災害: \url{http://weather.yahoo.co.jp/weather/jp/20/4810/20201.html}}
\bibitem{hoge3}{本文で参照していない文献はいれない.}
\bibitem{hoge4}{本文での登場順に参考文献リストに載せる.}
\bibitem{hoge5}{URLだらけにならないように! そのURLは数年後にもアクセス可能?}
\bibitem{hoge6}{「丸写し」と「引用」「参照」は違う!}
\end{thebibliography}

%%%%%%%%%%%%%%%%%%%%%%%%%%%%%%%%%%%%%%%%%%%%%%%%%%%%%%%%%%%%%%%%%%%%%%
%   付録
%%%%%%%%%%%%%%%%%%%%%%%%%%%%%%%%%%%%%%%%%%%%%%%%%%%%%%%%%%%%%%%%%%%%%%
\appendix
\chapter{おまけ}
本編に入れると冗長になる式変形や,話の細かいところは
ここに入れる.研究に利用したソフトウェアのインストール方法や
コンパイル方法なども詳細に書いておくと親切であろう.

\end{document}
